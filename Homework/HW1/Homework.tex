\documentclass{article}
\usepackage{amsmath}
\usepackage{amsfonts}
\usepackage{amsthm}
\usepackage{tikz}
\usepackage{algorithm2e}
\usepackage{fancyhdr}
\usepackage{extramarks}

%
% Homework Details
%   - Title
%   - Due date
%   - Class
%   - Section/Time
%   - Instructor
%   - Author
%

\newcommand{\hmwkTitle}{Homework\ \#1}
\newcommand{\hmwkDueDate}{January 14, 2020}
\newcommand{\hmwkClass}{MECH7610: Advanced Dynamics}
\newcommand{\hmwkClassInstructor}{Dr. Flowers}
\newcommand{\hmwkAuthorName}{\textbf{Matthew Boler}}

%
% Homework Problem Environment
%
% This environment takes an optional argument. When given, it will adjust the
% problem counter. This is useful for when the problems given for your
% assignment aren't sequential. See the last 3 problems of this template for an
% example.
%

\setcounter{secnumdepth}{0}
\newcounter{partCounter}
\newcounter{homeworkProblemCounter}
\setcounter{homeworkProblemCounter}{1}
\nobreak\extramarks{Problem \arabic{homeworkProblemCounter}}{}\nobreak{}

\newcommand{\enterProblemHeader}[1]{
    \nobreak\extramarks{}{Problem \arabic{#1} continued on next page\ldots}\nobreak{}
    \nobreak\extramarks{Problem \arabic{#1} (continued)}{Problem \arabic{#1} continued on next page\ldots}\nobreak{}
}

\newcommand{\exitProblemHeader}[1]{
    \nobreak\extramarks{Problem \arabic{#1} (continued)}{Problem \arabic{#1} continued on next page\ldots}\nobreak{}
    \stepcounter{#1}
    \nobreak\extramarks{Problem \arabic{#1}}{}\nobreak{}
}

\newenvironment{homeworkProblem}[1][-1]{
    \ifnum#1>0
        \setcounter{homeworkProblemCounter}{#1}
    \fi
    \section{Problem \arabic{homeworkProblemCounter}}
    \setcounter{partCounter}{1}
    \enterProblemHeader{homeworkProblemCounter}
}{
    \exitProblemHeader{homeworkProblemCounter}
}

%
% Basic Document Settings
%

\topmargin=-0.45in
\evensidemargin=0in
\oddsidemargin=0in
\textwidth=6.5in
\textheight=9.0in
\headsep=0.25in

\linespread{1.1}
\pagestyle{fancy}
\lhead{\hmwkAuthorName}
\chead{\hmwkClass: \hmwkTitle}
\rhead{\firstxmark}
\lfoot{\lastxmark}
\cfoot{\thepage}

\title{
    \vspace{2in}
    \textmd{\textbf{\hmwkClass:\ \hmwkTitle}}\\
    \normalsize\vspace{0.1in}\small{Due\ on\ \hmwkDueDate}\\
    \vspace{3in}
}

\author{\hmwkAuthorName}
\date{}


\begin{document}

\maketitle
\pagebreak

\begin{homeworkProblem}
    Determine the number of degrees of freedom of a uniform rod with length $L$ and mass $m$ fixed in the vertical plane by a pin through one end.

    \textbf{Solution}

    The system can be fully defined by the angle the rod makes with the horizonal, so the system has one degree of freedom.

\end{homeworkProblem}
\pagebreak

\begin{homeworkProblem}
    Write expressions for:
    \begin{enumerate}
        \item Kinetic energy of the system
        \item Potential energy of the system
    \end{enumerate}

    \textbf{Solution}

    \textbf{Part One}

    The kinetic energy of the system consists of a purely rotational component defined by:
    \begin{align*}
        T = \frac{1}{2}I_y \omega_y^2
    \end{align*}

    The potential energy of the system is limited to the energy due to gravity, equal to:
    \begin{align*}
        U &= mg\frac{L}{2}(1-\cos(\theta))
    \end{align*}
    where $z$ is the height of the center of mass above the center of mass of the earth.

\end{homeworkProblem}
\pagebreak

\begin{homeworkProblem}
    Draw the free body diagram for the system.

    \textbf{Solution}
    
\end{homeworkProblem}
\pagebreak

\begin{homeworkProblem}
    Formulate the equations of motion using Newton's Second Law.

    \textbf{Solution}
    Our equation of motion is:
    \begin{align*}
        I_y\ddot{\theta} &= \Sigma \tau_y = -mg\frac{L}{2}\cos(\theta) \\
    \end{align*}
\end{homeworkProblem}
\pagebreak

\begin{homeworkProblem}
    Formulate the equations of motion using Lagrange's Equations.

    \textbf{Solution}
    We form our Lagrangian:
    \begin{align*}
        L &= \frac{1}{2}I_y\dot{\theta}^2 - V(\theta) \\
        &= \frac{1}{2}I_y\dot{\theta}^2 - mg\frac{L}{2}(1-\cos\theta)
    \end{align*}
    Taking our Euler-Langrange derivative and setting the sides equal to each other, we get:
    \begin{align*}
        I_y\ddot{\theta} &= -\nabla V\\
        &= -mg\frac{L}{2}(1-\cos(\theta))
    \end{align*}
    \begin{align*}
        \frac{\partial L}{\partial \dot{\theta}} &= I_y \dot{\theta} \\
        \frac{d}{dt} (\frac{\partial L}{\partial \theta}) &= -mg\frac{L}{2}\sin(\theta)
    \end{align*}
    \begin{align*}
        \frac{\partial L}{\partial \dot{\theta}} - \frac{d}{dt} (\frac{\partial L}{\partial \theta}) &= I_y \ddot{\theta} + mg\frac{L}{2}\sin(\theta) = 0
    \end{align*}
\end{homeworkProblem}

    
\end{document}